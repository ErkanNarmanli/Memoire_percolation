\section{FiniteSize}
Dans cette partie on montre l'existence de clusters avec une bonne probabilité pour relier des parties de~$\Sk$ en vue d'une renormalisation. Pour cela, on introduit une suite~$\suite{u}$ telle que~$u_n \leqslant \frac{n}{3}$ et
	\[
		\Pp{\relie*[B_n]{B_{u_n}}{\partial B_n}} \underset{n \to \infty}{\to} 1.
	\] 
	On note~$S_n = B_{u_n}$ pour s'épargner une écriture trop lourde et on pose enfin
	\[
		\mathcal{E}_n( \alpha, \beta ) = \{\relie[B_n]{S_n}{\{n\} \times [ \alpha ; \beta ]} \}.
	\]
	\subsection{Un premier lemme}
		On commence par montrer le lemme suivant
		\begin{lem}\label{lem:collagesElem} 
			Il existe deux suites~$\suite{\alpha}$ et~$\suite{y}$ dans~$[0; n]$ telles que
			\begin{align*}
				  &\Pp{\mathcal{E}_n(\alpha_n, n)} \xrightarrow[n\to\infty]{} 1 \\ 
				  &\Pp{\mathcal{E}_n(y_n - \alpha_n/4, y_n + \alpha_n/4)} \xrightarrow[n \to \infty]{} 1.
			\end{align*}
		\end{lem}
		Dans la preuve de ce lemme ainsi que dans la suite de l'article, on utilisera librement l'inégalité de Harris-FKG aussi bien sous sa forme classique que sous la forme suivante qui en est une conséquence immédiate. %Bibliographie 
		\begin{lem}[square root trick]\label{lem:HarrisFKG}
			Soient~$A_1, A_2, \ldots, A_n$ des événements croissants ; alors
			\[
				\max_{1 \leq i \leq n} \Pp{A_i} \geq 1 - (1 - \Pp{A_1 \cup A_2 \cup \ldots A_n})^{1/n}. 
			\]
		\end{lem}
		
		JOLI DESSINS ICI %Will be done

		On passe maintenant à la démonstration du lemme~\ref{lem:collagesElem}.
		\begin{dem}
			En appliquant Harris-FKG à huit événements du type~$\mathcal{E}_n(\alpha, \beta)$, on obtient grâce à l'invariance par symétrie et rotation, que
			\[
				\Pp{\mathcal{E}_n(0,n)} \geq 1 - \left(1 - \Pp{\relie[B_n]{S_n}{\partial B_n}} \right)^{1/8}.
			\]
			D'où,~$\Pp{\mathcal{E}_n(0, n)}$ tend vers l'infini lorsque~$n$ tend vers l'infini. Sachant cela, on peut décomposer~$\mathcal{E}_n(0, n)$ de la manière suivante 
			\[
				\mathcal{E}_n(0, n) = \mathcal{E}_n(0, \alpha) \cup \mathcal{E}_n(\alpha + 1, n).
			\]
			Il suffit donc de choisir pour chaque~$n$ un~$\alpha$ assez grand pour que la deuxième condition du lemme soit vérifiée mais suffisamment éloigné de~$n$ pour que la première condition soit vérifiée. En effet, en appliquant à nouveau Harris-FKG, on obtient
			\[
				\max(\Pp{\mathcal{E}_n(0, 0)}, \Pp{\mathcal{E}_n(1, n)}) \xrightarrow[n \to \infty] 1.
			\]
			Or la probabilité de l'événement~$\mathcal{E}_n(0,0)$ est bornée uniformément par rapport à~$n$ par la constante~$1 - (1-p)^{3(k+1)}$ donc à partir d'un certain rang, la probabilité de~$\mathcal{E}_n(0,0)$ est inférieure à la probabilité de~$\mathcal{E}_n(1, n)}$ par une application de Harris-FKG. On obtient de la même manière que la probabilité de~$\mathcal{E}_n(0, n-1)$ est inférieure à celle de~$\mathcal{E}_n(n, n)$ à partir d'un certain rang. Cela permet de poser
			\[
				\alpha_n = \max \{ \alpha < n, \Pp{\mathcal{E}_n(0, \alpha - 1)} < \Pp{\mathcal{E}_n(\alpha, n)} \}.
			\]
			Ce choix de~$\alpha_n$ permet, en appliquant deux fois Harris-FKG, d'obtenir
			\[
				\Pp{\mathcal{E}_n(0, \alpha_n)}, \Pp{\mathcal{E}_n(\alpha_n, n)} \geq 1 - \sqrt{1  - \Pp{\mathcal{E}_n(0, n)}} \underset{n \to \infty}{\to} 1.
			\]
			Puis, pour choisir~$y_n$, on effectue la décomposition
			\[
				\mathcal{E}_n(0, \alpha_n) = \mathcal{E}_n(0, \alpha_n/2) \cup \mathcal{E}_n(\alpha_n/2, \alpha_n).
			\]
			Il suffit alors de choisir~$y_n$ égal à $\alpha_n/4$ ou~$3\alpha_n/4$ de sorte à maximiser la probabilité de~$\mathcal{E}_n(y_n - \alpha_n/4, y_n + \alpha_n/4)$. Par une dernière application de l'inégalité de Harris-FKG, on obtient enfin
			\[
				\Pp{\mathcal{E}_n(y_n - \alpha_n/4, y_n + \alpha_n/4)} \geq 1 - \sqrt{1 - \Pp{\mathcal{E}_n(0, \alpha_n)}}
			\]
			Ce qui permet de conclure, par l'assertion précédente, que ces choix de~$y_n$ et~$\alpha_n$ conviennent.
		\end{dem}

		%Globalement, sans regarder le pdf, je dirai qu'il y a *beaucoup* trop de maths en ligne et hors ligne
		%De toute façon il faudra qu'on fasse une relecture papier à la fin, mais il faudrait quand même réduire par mal 
		%la quantité de maths hors ligne.
		%Question : t'as check les badness sortie de ligne à droite ?
		%Remarque : c'est super bien indenté :)

