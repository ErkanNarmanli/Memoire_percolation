\section{Le lemme gluant}
	\subsection{Énoncé}
		On démontre à présent le \emph{lemme gluant}, dont le nom est une traduction très approximative du \emph{gluing lemma}, le lemme de collage.
			
		\begin{lem}[dit gluant]\label{lem:gluant}
			Soit~$\epsilon$ un réel strictement positif ; alors il existe un réel~$\delta$ strictement positif tel que pour tout entier naturel~$n$ on ait :
			\[
				\Pp{\relie[B_{3n}]{S_{3n}}{Z_n},
					\relie[B_n']{S_n'}{Y_n^-},
					\relie[B_n']{S_n'}{Y_n^+}}
				\geq 1 - \delta
			\]
			implique
			\[
				\Pp{\relie[B_{3n}\cup B_n']{S_{3n}}{S_n'}}
				\geq 1 - \epsilon.
			\]
		\end{lem}
		On se rend compte que le lemme gluant est trivial pour~$k=0$, on peut dès lors supposer que~$k$ est un entier strictement positif. Pour démontrer le lemme gluant on a déjà besoin démontrer le lemme~\ref{lem:interGluant}.
	
	\subsection{Lemme intermédiaire}
		Le résultat intermédiaire suivant nous permet d'obtenir une estimation sur les probabilités de deux événements sous réserve que l'on dispose d'une application entre ces deux évènements qui vérifie de bonnes propriétés.
		\begin{lem}\label{lem:interGluant}
			Soient~$s,t$ deux entiers naturels strictement positifs, soient~$\mathcal{A},\mathcal{B}$ deux événements dans la tribu~$\mathcal{P}(B_n)$ pour un certain cylindre~$B_n$, et~$\Phi : \mathcal{A} \rightarrow \mathcal{P}(\mathcal{B})$ une application telle que :
			\begin{enumerate}
				\item pour tout~$\omega$ dans~$\mathcal{A}$, on ait~$\#\Phi(\omega)\geq t$ ; \label{item:interGluant:a}
				\item pour tout~$\omega'$ dans~$\mathcal{B}$, il existe un sous-ensemble fini~$S$ de~$\Sk$ qui soit de cardinal au plus~ $s$ et tel que l'on ait l'inclusion~$\{\omega\in\mathcal{A},\omega'\in\Phi(\omega)\} \subset \{\omega\in\mathcal{A},\omega_{|S^c}=\omega_{|S^c}'\}$. \label{item:interGluant:b}
			\end{enumerate}
			Alors , on a la majoration :
			\[
				\Pp{\mathcal{A}} \leq \frac{1}{t}
					\left(\frac{2}{\min(p,1-p)}\right)^s
					\Pp{\mathcal{B}}.
			\]
		\end{lem}
		\begin{dem}
			Dans un premier temps, et comme on est dans la tribu discrète, on peut écrire la décomposition suivante :
			\begin{equation}\label{eq:interGluant:1}
				\sum_{\omega\in\mathcal{A}}\Pp{\Phi(\omega)} = \sum_{\omega\in\mathcal{A}}\sum_{\omega'\in\Phi(\omega)}\Pp{\omega'}.
			\end{equation}
			Dans la seconde somme du terme de droite, on a~$\omega'$ dans~$\Phi(\omega)$, donc par la condition~(\ref{item:interGluant:a}) du lemme on a~$\omega_{|S^c}=\omega_{|S^c}'$, où~$S$ est un sous ensemble fini de~$\Sk$, de cardinal au plus~$s$. Dès lors, on peut écrire~$\PpSP(\omega) = \PpSP(\omega_{|S^c})\cdot\PpSP(\omega_{|S}')$, puisque la configuration sur~$S^c$ et~$S$ sont indépendantes. Et comme~$\PpSP(\omega_{|S}')$ est un produit d'au plus~$s$ facteurs~$p$ ou~$1-p$, on a la majoration~$\PpSP(\omega') \leq \PpSP(\omega)\min(p,1-p)^s$ ; en combinant cette majoration à l'équation~\ref{eq:interGluant:1}, on obtient :
			\begin{equation}\label{eq:interGluant:2}
				\sum_{\omega\in\mathcal{A}} \Pp{\omega}
					\leq
					\frac{1}{t\cdot\min(p,1-p)^s}
					\sum_{\omega\in\mathcal{A}}\Pp{\Phi(\omega)}.
			\end{equation}
			On peut encore décomposer la somme dans le terme de droite terme à terme dans~$\Phi(\omega)$, comme précédemment, puis inverser les deux sommes pour obtenir :
			\begin{equation}\label{eq:interGluant:3}
				\sum_{\omega\in\mathcal{A}} \Pp{\omega}
					\leq
					\frac{1}{t\cdot\min(p,1-p)^s}
					\sum_{\omega'\in\mathcal{B}}
					\sum_{\substack{\omega\in\mathcal{A}\\ \omega'\in\Phi(\omega)}}\Pp{\Phi(\omega')}.				
			\end{equation}
			On peut maintenant majorer le nombre d'éléments dans la deuxième somme à droite grâce à la condition~(\ref{item:interGluant:b}) du lemme. La somme contient moins d'élément qu'il n'y a de configuration qui coïncide avec~$\omega$ sur~$S^c$, à savoir~$2^s$ ; il suffit ensuite de reporter cette majoration dans l'équation~\ref{eq:interGluant:3} pour obtenir le résultat souhaité.    
		\end{dem}
	
	\subsection{Preuve du lemme gluant}
		On va maintenant donner la démonstration du lemme gluant (lemme~\ref{lem:gluant}).
		\paragraph{Relation d'ordre sur les chemins}
			Soit~$\omega$ une configuration, on va définir une relation d'ordre totale~$\leq$ sur l'ensemble des chemins auto-évitant qui vont de~$\SnB$ à~$\ZnB$. Pour ce faire, on défini déjà deux relations d'ordre comme suit :
			\begin{itemize}
				\item pour tout sommet de~$\mathbb{Z}^2$, on définit une relation d'ordre totale~$\preceq$ sur les arrêtes qui partent dudit sommet, de telle sorte que la relation soit invariante par translation de~$\mathbb{Z}^2$ ;
				\item on prend~$\ll$ une relation d'ordre totale sur les arrêtes de~$\Sk$.
			\end{itemize}
			Par la suite, si~$\gamma = \left(\gamma_i\right)_{1\leq i\leq r}$ et~$\gamma'=\left(\gamma'_i\right)_{1\leq i\leq r'}$ sont deux chemins auto-évitants qui vont de~$\SnB$ à~$\ZnB$ alors on notera~$\gamma <\gamma'$ lorsqu'au moins l'une des conditions suivantes sont satisfaites :
			\begin{itemize}
				\item le chemin~$\gamma$ est un sous chemin de~$\gamma'$, c'est à dire que~$r<r'$ et que~$\gamma = \left(\gamma_i'\right)_{1\leq i\leq r}$, cela traite le cas où les chemins ne se séparent qu'une fois arrivés à~$\ZnB$ ;
				\item on peut comparer strictement les premiers sommets des deux chemins avec~$\gamma_0 \ll \gamma_0'$, ce qui traite le cas où les deux chemins sont différents dès la sortie de~$\SnB$ ;
				\item il existe une entier~$k$ strictement plus petit que~$r$ et~$r'$ tel que~$\gamma_i=\gamma_i'$ pour tout entier~$i$ comprit entre~$0$ et~$k$ et tel que~$(\gamma_k,\gamma_{k+1}) \prec (\gamma_k',\gamma_{k+1}')$, c'est à dire que les chemins coïncident jusqu'à la~$k$-ième arrête et que l'on peut comparer les arrêtes issues de~$\gamma_k$.
		\end{itemize}
		\paragraph{Chemin minimal}
			Soit~$\omega$ une configuration, comme~$\leq$ est une relation totale sur l'ensemble fini des chemins auto-évitant entre~$\SnB$ et~$\ZnB$, on peut définir~$\Cmin$ comme étant l'élément minimal pour cette relation dans~$\omega$. Dès lors on définit l'ensemble~$U(\omega)$ comme étant l'ensemble des éléments de~$B_n'$ vérifiant les deux propriétés suivantes :
			\begin{enumerate}[label=P\arabic*.]
				\item\label{item:P1} La droite~$\overline{\{z\}}$ et le support de~$\Cmin$ sont d'intersection non vide.
				\item\label{item:P2} Il existe un chemin~$\pi$ qui relie les cylindres~$\overline{z+B_1}$ et~$\SnB[n]'$ ; et tel que la distance entre les projections canoniques de~$\pi$ et de~$\Cmin$ dans~$\mathbb{Z}^2$ est exactement~$1$.
			\end{enumerate}
			Dans cette définition, la propriété~\ref{item:P1} signifie que tout élément~$z$ de~$U(\omega)$ doit être sur la projection du chemin minimal~$\Cmin$ alors que la propriété deux demande qu'il existe au moins un point au dessus de~$z$ qui soit presque relié ou presque à~$\SnB[n]'$ par un chemin qui ne passe jamais au dessus ou au dessous de~$\Cmin$.
		
		\boldmath
		\paragraph{L'évènement~$\mathcal{X}$} Pour la suite de la démonstration, on définit l'évènement suivant :
		\unboldmath
			\[
				\mathcal{X} =
				\left\{
					\relie[B_{3n}]{S_{3n}}{Z_n},
					\relie[B_n']{S_n'}{Y_n^-},
					\relie[B_n']{S_n'}{Y_n^+}
				\right\}
				\cup
				\left\{
					\relie[B_{3n}\cup B_n']{S_{3n}}{S_n'}
				\right\}^c
			\]
			Le premier ensemble dans la définition de~$\mathcal{X}$ est l'évènement que l'on suppose avec une probabilité d'au moins~$1-\delta$ alors que le reste est le complémentaire de l'évènement que l'on souhaite avoir avec probabilité au moins~$1-\epsilon$. Pour démontrer le lemme gluant~(lemme~\ref{lem:gluant}), on est donc ramené à montrer que la probabilité~$\PpSP(\mathcal{X})$ est très petite lorsque la probabilité~$\PpSP(\relie[B_{3n}]{S_{3n}}{Z_n}, \relie[B_n']{S_n'}{Y_n^-}, \relie[B_n']{S_n'}{Y_n^+})$ est proche de~$1$. Pour ce faire, on va séparer le problème en deux, en fonction de la taille de~$U(\omega)$. On dispose pour cela des deux proposition suivantes.
			\begin{prop}\label{prop:Upetit}
				Soient~$t$ un entier naturel non nul et~$\epsilon$ un réel strictement positif ; alors il existe~$\delta$ un réel strictement positif tel que :
				\[
					\Pp{\relie[B_n']{S_n'}{Y_n^-}, \relie[B_n']{S_n'}{Y_n^+}} > 1 - \delta
				\]
				implique
				\[
					\Pp{\mathcal{X}\cap\{\#U<t\}} \leq \epsilon.
				\]
			\end{prop}
			\begin{dem}
i				Soit~$\omega$ une configuration dans~$\mathcal{X} \cap \{\# U < t\}$, on définit la configuration~$\omega'$ à partir de~$\omega$ en y retirant les arrêtes de la forme~$\{u,v\}$ avec~$u$ dans la droite~$\overline{\{z\}}$ pour~$z$ dans~$U(z)$ et~$v$ relié à~$\SnB[n]'$.
				On a donc construit~$\omega'$ de telle sorte qu'il ne puisse pas contenir dans~$\BnB '$ à la fois un chemin reliant~$\SnB[n]'$ et~$Y_n^-$ et un chemin reliant~$\SnB[n]'$ et~$Y_n^+$ ; car sinon~$\omega$ contiendrait un chemin qui relierait~$S_{3n}$ à~$S_n'$ dans~$B_{3n}\cup B_n'$, ce qui contredirait~$\omega\in\mathcal{X}$.
			
				La construction que nous venons de concevoir nous permet de définir l'application~$\Phi$ comme suit :
				\begin{align*}
					\Phi : \mathcal{X}\cup\{\# U < t\} 	& \longrightarrow 	
											\mathcal{P}\left(
												\{\relie[B_n']{S_n'}{Y_n^-}
												\relie[B_n']{S_n'}{Y_n^+}\}^c
											\right)\\
						\omega				& \longmapsto 		\{\omega'\}
				\end{align*}	
				Les propriétés~\ref{item:P1} et~\ref{item:P2} nous assurent que l'on a retiré aucune arrête de~$\Cmin$, on a donc~$\Cmin[\omega'] = \Cmin$. Comme on retire des arrête que dans des cylindres~$\overline{z+B_1}$ et que le projeté des chemins~$\pi$ de la propriété~\ref{item:P2} sont tous à distance au moins~$1$ de la projection de~$\Cmin$, on a aussi~$U(\omega) = U(\omega')$.

				De plus~$\Phi^{<-1>}(\omega')$ contient seulement des configurations qui sont égales à~$\omega'$ sauf peut être en des arrêtes adjacentes à~$U(\omega)$. Comme~$U(\omega)$ contient au plus~$t$ élément, il y a au plus~$(5k+1)t$ arrêtes adjacentes à~$U(\omega)$. On peut alors appliquer le lemme~\ref{lem:interGluant} à~$\Phi$ avec~$t'=1$ et~$s=(5k+1)t$ qui nous donne la majoration :
				\[
						\Pp{\mathcal{X}\cup\{\#U < t\}}
					\leq
						\left(
							\frac{2}{\min(p,1-p)}
						\right)^{(5k+1)t}
						\left( 1 - \Pp{\relie[B_n']{S_n'}{Y_n^-}, \relie[B_n']{S_n'}{Y_n^+}}\right).
				\]
				Il suffit alors de prendre~$\delta$ assez petit pour achever la démonstration.
			\end{dem}
			\begin{prop}\label{prop:Ugrand}
				Soit~$\epsilon$ un réel strictement positif ; alors il existe un entier~$t$ strictement positif tel que l'on ait la majoration :
				\[
						\Pp{\mathcal{X}\cap\{\#U \geq t\}} 
					\leq 
						\epsilon
						\cdot
						\Pp{\relie[B_{3n}\cup B_n']{S_{3n}}{S_n'}}.
				\]
			\end{prop}
			\begin{dem}
				[TODO]
			\end{dem}
			On peut maintenant achever la démonstration du lemme gluant~(lemme~\ref{lem:gluant}). Fixons~$\epsilon$ un réel strictement positif, alors d'après la proposition~\ref{prop:Ugrand}, il existe~$t$ un entier naturel non nul tel que :
			\begin{equation}\label{eq:finalGluant:1}
					\Pp{\mathcal{X}\cap\{\#U \geq t\}} 
				\leq 
					\frac{\epsilon}{2}
					\cdot
					\Pp{\relie[B_{3n}\cup B_n']{S_{3n}}{S_n'}}.			
			\end{equation}
			On peut ensuite appliquer la proposition~\ref{prop:Upetit} à~$\epsilon/2$ à~$t$ : il existe un réel~$\delta$ strictement positif tel que lorsque la probabilité de l'événement~$\{\relie[B_n']{S_n'}{Y_n^-}\}\cap\{\relie[B_n']{S_n'}{Y_n^+}\}$ est plus grande que~$1-\delta$ alors on a :
			\begin{equation}\label{eq:finalGluant:2}
				\Pp{\mathcal{X}\cap\{\#U\leq t\}}
				\leq
				\frac{\epsilon}{2}
				\cdot %ponctuation
			\end{equation}
			Dès lors lorsque la probabilité de~$\{\relie[B_{3n}\cup B_n']{S_{3n}}{S_n'}\} \cap \{\relie[B_n']{S_n'}{Y_n^-}\} \cap \{\relie[B_n']{S_n'}{Y_n^+}\}$ est plus grande que~$1-\delta$, on peut utiliser les équations~\ref{eq:finalGluant:1} et~\ref{eq:finalGluant:2} pour obtenir~$\PpSP(\mathcal{X}) \leq \epsilon$ ; ce qui achève la démonstration du lemme gluant.

