\section{Le lemme gluant}
	\subsection{Énoncé}
		On démontre à présent le \emph{lemme gluant}, dont le nom est une traduction très approximative du \emph{gluing lemma}, le lemme de collage. La figure~\ref{fig:gluant} fournit une illustration dudit lemme.
			
		\begin{lem}[dit gluant]\label{lem:gluant}
			Soit~$\epsilon$ un réel strictement positif ; alors il existe un réel~$\delta$ strictement positif tel que pour tout entier naturel~$n$ on ait :
			\[
				\Pp{\relie[B_{3n}]{S_{3n}}{Z_n},
					\relie[B_n']{S_n'}{Y_n^-},
					\relie[B_n']{S_n'}{Y_n^+}}
				\geq 1 - \delta
			\]
			implique
			\[
				\Pp{\relie[B_{3n}\cup B_n']{S_{3n}}{S_n'}}
				\geq 1 - \epsilon.
			\]
		\end{lem}
		\begin{figure}[h]
			\begin{center}
			\begin{quartCarre}
				%Courbes de Bézier 1
				\draw 	[thick]	(0,2) 	.. controls (0,6) 		and (-3,10) .. (3,10);
				\draw 	[thick]	(3,10)	.. controls (6,10) 		and (5,7) 	.. (7,5);
				\draw	[thick]	(7,5)		.. controls (9,3)		and (13,6) 	.. (15,7.5);
				%Courbe de Bézier 2	
				\draw	[thick, color=gray!85]	(9,8)		.. controls (8,8)		and (8,8) 	.. (7,5);
				\draw	[thick, color=gray!85]	(7,5)		.. controls (6,2)		and (13,5)	.. (15, 5);
				%Courbe de Bézier 3
				\draw 	[thick, color=gray!85]	(10.5,9)	.. controls (9.5,12)	and (12,14)	.. (15,11);
				%Plages
				\fill [color=gray!40]	(14.5,3)	rectangle (15.5,6);
				\fill [color=gray!40]	(14.5,10)	rectangle (15.5,13);
				\fill [color=gray!8]	(14.5,6)	rectangle (15.5,10);
				\fill [pattern=north east lines]	(14.5,6)	rectangle (15.5,10);
				%Text + barres de séparation
				\draw [thick]	(5,3) 		-- (16,3);
				\draw [thick]	(5,13)		-- (16,13);
				\draw 			(14.5, 6)	-- (16,6);
				\draw 			(14.5, 6)	-- (16,6);
				\draw 			(14.5, 10)	-- (16,10);
				\path	[right]	(15.5, 8)		node	{$Z_n$}	;
				\path	[right]	(15.5, 4.5)		node	{$Y_n^+$}	;
				\path	[right]	(15.5, 11.5)	node	{$Y_n^-$}	;
			\end{quartCarre}
			\quad\quad
			\begin{quartCarre}
				%Courbes de Bézier 1
				\draw 	[thick]	(0,2) 	.. controls (0,6) 		and (-3,10) .. (3,10);
				\draw 	[thick]	(3,10)	.. controls (6,10) 		and (5,7) 	.. (7,5);
				%Courbe de Bézier 2	
				\draw	[thick]	(9,8)		.. controls (8,8)		and (8,8) 	.. (7,5);
				%Plages
				\fill [color=gray!40]	(14.5,3)	rectangle (15.5,6);
				\fill [color=gray!40]	(14.5,10)	rectangle (15.5,13);
				\fill [color=gray!8]	(14.5,6)	rectangle (15.5,10);
				\fill [pattern=north east lines]	(14.5,6)	rectangle (15.5,10);
				%Text + barres de séparation
				\draw [thick]	(5,3) 		-- (16,3);
				\draw [thick]	(5,13)		-- (16,13);
				\draw 			(14.5, 6)	-- (16,6);
				\draw 			(14.5, 6)	-- (16,6);
				\draw 			(14.5, 10)	-- (16,10);
				\path	[right]	(15.5, 8)		node	{$Z_n$}	;
				\path	[right]	(15.5, 4.5)		node	{$Y_n^+$}	;
				\path	[right]	(15.5, 11.5)	node	{$Y_n^-$}	;
			\end{quartCarre}
			\end{center}
			\caption{Le lemme gluant affirme que si l'événement représenté à gauche a une probabilité supérieure à~$1-\delta$ alors on peut avoir événement de droite avec une probabilité~$1-\epsilon$.}
			\label{fig:gluant}
		\end{figure}
		Le résultat est trivial pour~$k=0$, on peut dès lors supposer que~$k$ est un entier strictement positif. Pour démontrer le lemme gluant on a d'abord besoin démontrer le lemme~\ref{lem:interGluant}.
	
	\subsection{Lemme intermédiaire}
		Le résultat intermédiaire suivant nous permet d'obtenir une estimation sur les probabilités de deux événements~$\mathcal{A}$ et~$\mathcal{B}$ sous réserve que l'on dispose d'une application entre~$\mathcal{A}$ et~$\mathcal{P}(\mathcal{B})$ qui vérifie de bonnes propriétés.
		\begin{lem}\label{lem:interGluant}
			Soient~$s,t$ deux entiers naturels strictement positifs, soient~$\mathcal{A},\mathcal{B}$ deux événements dans la tribu~$\mathcal{P}(B_n)$ pour un certain cylindre~$B_n$, et~$\Phi : \mathcal{A} \rightarrow \mathcal{P}(\mathcal{B})$ une application telle que :
			\begin{enumerate}
				\item pour tout~$\omega$ dans~$\mathcal{A}$, on ait~$\#\Phi(\omega)\geq t$ ; \label{item:interGluant:a}
				\item pour tout~$\omega'$ dans~$\mathcal{B}$, il existe un sous-ensemble fini~$S$ de~$\Sk$ qui soit de cardinal au plus~$s$ et tel que l'on ait l'inclusion~$\{\omega\in\mathcal{A},\omega'\in\Phi(\omega)\} \subset \{\omega\in\mathcal{A},\omega_{|S^c}=\omega_{|S^c}'\}$. \label{item:interGluant:b}
			\end{enumerate}
			Alors ; on a la majoration :
			\[
				\Pp{\mathcal{A}} \leq \frac{1}{t}
					\left(\frac{2}{\min(p,1-p)}\right)^s
					\Pp{\mathcal{B}}.
			\]
		\end{lem}
		\begin{dem}
			Dans un premier temps, et comme on est dans la tribu discrète, on peut écrire la décomposition suivante :
			\begin{equation}\label{eq:interGluant:1}
				\sum_{\omega\in\mathcal{A}}\Pp{\Phi(\omega)} = \sum_{\omega\in\mathcal{A}}\ \sum_{\omega'\in\Phi(\omega)}\Pp{\omega'}.
			\end{equation}
			Dans la seconde somme du terme de droite, on a~$\omega'$ dans~$\Phi(\omega)$, donc par la condition~(\ref{item:interGluant:a}) du lemme on a~$\omega_{|S^c}=\omega_{|S^c}'$, où~$S$ est un sous ensemble fini de~$\Sk$, de cardinal au plus~$s$. Dès lors, on peut écrire~$\PpSP{}[\omega'] = \PpSP{}[\omega_{|S^c}]\cdot\PpSP{}[\omega_{|S}']$, puisque la configuration sur~$S^c$ et~$S$ sont indépendantes. Et comme~$\PpSP{}[\omega_{|S}']$ est un produit d'au plus~$s$ facteurs~$p$ ou~$1-p$, on a la majoration~$\PpSP{}[\omega'] \geq \PpSP{}[\omega]\min(p,1-p)^s$ ; en combinant cette majoration à l'équation~\eqref{eq:interGluant:1}, on obtient :
			\begin{equation}\label{eq:interGluant:2}
					\sum_{\omega\in\mathcal{A}}\Pp{\Phi(\omega)}
				\geq
					\min(p,1-p)^s
					\sum_{\omega\in\mathcal{A}} \sum_{\omega'\in\Phi(\omega)} 
						\Pp{\omega}
				\geq
					t\cdot\min(p,1-p)^s
					\sum_{\omega\in\mathcal{A}}\Pp{\omega}.
			\end{equation}
			On peut encore décomposer la somme dans le terme de gauche sur~$\Phi(\omega)$, comme précédemment, puis inverser les deux sommes pour obtenir :
			\begin{equation}\label{eq:interGluant:3}
					\sum_{\omega'\in\mathcal{B}}
					\sum_{\substack{\omega\in\mathcal{A}\\ \omega'\in\Phi(\omega)}}\Pp{\Phi(\omega')}
				\geq
					t\cdot\min(p,1-p)^s
.					\sum_{\omega\in\mathcal{A}} \Pp{\omega}
			\end{equation}
			On peut maintenant majorer le nombre d'éléments dans la deuxième somme à gauche grâce à la condition~(\ref{item:interGluant:b}) du lemme. La somme contient moins d'éléments qu'il n'y a de configurations qui coïncident avec~$\omega$ sur~$S^c$, à savoir~$2^s$ ; il suffit ensuite de reporter cette majoration dans l'équation~\eqref{eq:interGluant:3} pour obtenir le résultat souhaité.
		\end{dem}
	
	\subsection{Preuve du lemme gluant}
		On va maintenant donner la démonstration du lemme gluant (lemme~\ref{lem:gluant}).
		\paragraph{Relation d'ordre sur les chemins}
			Soit~$\omega$ une configuration, on va définir une relation d'ordre totale~$\leq$ sur l'ensemble des chemins auto-évitant qui vont de~$\SnB$ à~$\ZnB$. Pour ce faire, on définit déjà deux relations d'ordre comme suit :
			\begin{itemize}
				\item pour tout sommet de~$\mathbb{Z}^2$, on définit une relation d'ordre totale~$\preceq$ \marginnote{$\preceq$} sur les arêtes qui partent dudit sommet, de telle sorte que la relation soit invariante par translation de~$\mathbb{Z}^2$ ;
				\item on prend~$\ll$ \marginnote{$\ll$} une relation d'ordre totale sur les arêtes de~$\Sk$.
			\end{itemize}
			Par la suite, si~$\gamma = \left(\gamma_i\right)_{1\leq i\leq r}$ et~$\gamma'=\left(\gamma'_i\right)_{1\leq i\leq r'}$ sont deux chemins auto-évitants qui vont de~$\SnB$ à~$\ZnB$ alors on notera~$\gamma <\gamma'$ \marginnote{$\leq$} lorsqu'au moins l'une des conditions suivantes sont satisfaites :
			\begin{itemize}
				\item le chemin~$\gamma$ est un sous chemin strict de~$\gamma'$, c'est à dire que~$r<r'$ et que~$\gamma = \left(\gamma_i'\right)_{1\leq i\leq r}$, cela traite le cas où les chemins ne se séparent qu'une fois arrivés à~$\ZnB$ ;
				\item on peut comparer strictement les premiers sommets des deux chemins avec~$\gamma_0 \ll \gamma_0'$, ce qui traite le cas où les deux chemins sont différents dès la sortie de~$\SnB$ ;
				\item il existe un entier~$k$ strictement plus petit que~$r$ et~$r'$ tel que~$\gamma_i=\gamma_i'$ pour tout entier~$i$ compris entre~$0$ et~$k$ et tel que~$(\gamma_k,\gamma_{k+1}) \prec (\gamma_k',\gamma_{k+1}')$, c'est à dire que les chemins coïncident jusqu'à la~$k$-ième arête et que l'on peut comparer les arêtes issues de~$\gamma_k$.
		\end{itemize}
		\paragraph{Chemin minimal}
			Soit~$\omega$ une configuration, comme~$\leq$ est une relation totale sur l'ensemble fini des chemins auto-évitants entre~$\SnB$ et~$\ZnB$, on peut définir~$\Cmin$ \marginnote{$\Cmin$} comme étant l'élément minimal pour cette relation dans~$\omega$. Dès lors, on définit~$U(\omega)$\marginnote{$U(\omega)$} comme étant l'ensemble des éléments de~$B_n'$ vérifiant les deux propriétés suivantes :
			\begin{enumerate}[label=\textbf{P\arabic*}., ref=P\arabic*]
				\item\label{item:P1} Le segment~$\overline{\{z\}}$ et le support de~$\Cmin$ sont d'intersection non vide.
				\item\label{item:P2} Il existe un chemin~$\pi$ qui relie les cylindres~$\overline{z+B_1}$ et~$\SnB[n]'$ ; et tel que la distance entre les projections canoniques de~$\pi$ et de~$\Cmin$ dans~$\mathbb{Z}^2$ soit exactement~$1$.
			\end{enumerate}
			Dans cette définition, la propriété~\ref{item:P1} signifie que tout élément~$z$ de~$U(\omega)$ doit être sur la projection du chemin minimal~$\Cmin$ alors que la propriété~\ref{item:P2} demande qu'il existe au moins un point au dessus de~$z$ qui soit presque relié ou presque à~$\SnB[n]'$ par un chemin qui ne passe jamais au dessus ou au dessous de~$\Cmin$.
		
		\boldmath
		\paragraph{L'évènement~$\mathcal{X}$}
		\unboldmath
		Pour la suite de la démonstration, on définit l'évènement suivant :
		
			\[ 	\marginnote{$\mathcal{X}$}
				\mathcal{X} =
				\left\{
					\relie[B_{3n}]{S_{3n}}{Z_n},
					\relie[B_n']{S_n'}{Y_n^-},
					\relie[B_n']{S_n'}{Y_n^+}
				\right\}
				\cup
				\left\{
					\relie[B_{3n}\cup B_n']{S_{3n}}{S_n'}
				\right\}^c
			\]
			Dans les hypothèses du lemme gluant, on suppose que le premier événement dans la définition de~$\mathcal{X}$ a une probabilité d'au moins~$1-\delta$ ; on veut montrer que complémentaire du second événement a une probabilité d'au moins~$1-\epsilon$. Pour démontrer le lemme gluant~(lemme~\ref{lem:gluant}), on est donc ramené à montrer que la probabilité~$\PpSP(\mathcal{X})$ est très petite lorsque la probabilité du premier terme dans son expression %~$\PpSP(\relie[B_{3n}]{S_{3n}}{Z_n}, \relie[B_n']{S_n'}{Y_n^-}, \relie[B_n']{S_n'}{Y_n^+})$
			est proche de~$1$. Pour ce faire, on va séparer le problème en deux, en fonction de la taille de~$U(\omega)$. On dispose pour cela des proposition~\ref{prop:Upetit} et~\ref{prop:Ugrand}.
			\begin{prop}\label{prop:Upetit}
				Soient~$t$ un entier naturel non nul et~$\epsilon$ un réel strictement positif ; alors il existe un réel~$\delta$ strictement positif tel que :
				\[
					\Pp{\relie[B_n']{S_n'}{Y_n^-}, \relie[B_n']{S_n'}{Y_n^+}} > 1 - \delta
						\quad\quad
						\Rightarrow
						\quad\quad
					\Pp{\mathcal{X}\cap\{\#U<t\}} \leq \epsilon.
				\]
			\end{prop}
			\begin{dem}
				Soit~$\omega$ une configuration dans~$\mathcal{X} \cap \{\# U < t\}$, on définit la configuration~$\omega'$ à partir de~$\omega$ de la façon suivante : on y retire les arêtes de la forme~$\{u,v\}$ avec~$u$ dans la droite~$\overline{\{z\}}$ pour un certain~$z$ dans~$U(\omega)$ et~$v$ relié à~$\SnB[n]'$.
				On a donc construit~$\omega'$ de telle sorte qu'il ne puisse pas contenir dans~$\BnB '$ à la fois un chemin reliant~$\SnB[n]'$ et~$Y_n^-$ et un chemin reliant~$\SnB[n]'$ et~$Y_n^+$ ; car sinon~$\omega$ contiendrait un chemin qui relierait~$S_{3n}$ à~$S_n'$ dans~$B_{3n}\cup B_n'$, ce qui contredirait~$\omega\in\mathcal{X}$.
			
				La construction que nous venons de présenter nous permet de définir l'application~$\Phi$ comme suit :
				\begin{align*}
					\Phi \marginnote{$\Phi$} : \mathcal{X}\cup\{\# U < t\} 	& \longrightarrow 	
											\mathcal{P}\left(
												\left\{\relie[B_n']{S_n'}{Y_n^-},
												\relie[B_n']{S_n'}{Y_n^+}\right\}^c
											\right)\\
						\omega				& \longmapsto 		\{\omega'\}
				\end{align*}	
				Les propriétés~\ref{item:P1} et~\ref{item:P2} nous assurent que l'on n'a retiré aucune arête de~$\Cmin$, on a donc~$\Cmin[\omega'] = \Cmin$. Comme on ne retire des arêtes que dans des cylindres de la forme~$\overline{z+B_1}$ et que le projeté des chemins~$\pi$ de la propriété~\ref{item:P2} sont tous à distance au moins~$1$ de la projection de~$\Cmin$, on a aussi~$U(\omega) = U(\omega')$.

				De plus~$\Phi^{-1}(\omega')$ contient seulement des configurations qui sont égales à~$\omega'$ sauf peut être en des arêtes adjacentes à~$U(\omega)$. Comme~$U(\omega)$ contient au plus~$t$ éléments, il y a au plus~$(5k+1)t$ arêtes adjacentes à~$U(\omega)$. On peut alors appliquer le lemme~\ref{lem:interGluant} à~$\Phi$ avec~$t'=1$ et~$s=(5k+1)t$ qui nous donne la majoration :
				\[
						\Pp{\mathcal{X}\cup\{\#U < t\}}
					\leq
						\left(
							\frac{2}{\min(p,1-p)}
						\right)^{(5k+1)t}
						\left( 1 - \Pp{\relie[B_n']{S_n'}{Y_n^-}, \relie[B_n']{S_n'}{Y_n^+}}\right).
				\]
				Il suffit alors de prendre~$\delta$ assez petit pour achever la démonstration.
			\end{dem}
			\begin{prop}\label{prop:Ugrand}
				Soit~$\epsilon$ un réel strictement positif ; alors il existe un entier~$t$ strictement positif tel que l'on ait la majoration :
				\[
						\Pp{\mathcal{X}\cap\{\# U \geq t\}} 
					\leq 
						\epsilon
						\cdot
						\Pp{\relie[B_{3n}\cup B_n']{S_{3n}}{S_n'}}.
				\]
			\end{prop}
			\begin{rem}
				Si~$z = (x_1, x_2, x_3)$ est un point de~$\Sk$ et si~$R$ est un entier naturel, on notera~$\overline{B_R}(z)$ \marginnote{$\overline{B_R}(z)$} à la place de~$\overline{(x_1, x_2) + B_R}$. Supposons maintenant~$R=2$, si~$u, v, w$ sont des voisins de~$z$ distincts deux à deux et si~$u', v', w'$ sont trois points distincts sur la frontière~$\partial \BRz$, alors il existe trois chemins auto-évitants disjoints dans~$\BRz\priv\{z\}$ qui relient respectivement~$u, v, w$ à~$u', v', w'$, puisque que l'on a supposé plus tôt que~$k$ est non nul. 
			\end{rem}
			\begin{dem}[de la proposition~\ref{prop:Ugrand}]
				Soient~$\omega$ une configuration dans~$\mathcal{X}\cap\{\# U \geq t\}$ et~$z$ un sommet dans~$U(\omega)$, on va construire la configuration~$\omega^{(z)}$ \marginnote{$\omega^{(z)}$} comme suit :
				\begin{enumerate}
					\item\label{item:cons:a} On prend~$u,v,w$ \marginnote{$u,v,w$} trois points distincts qui sont  à distance~$1$ de~$z$  et telle que l'on puisse comparer les arêtes avec~$(z,v)\prec (z,w)$.
					
					\item\label{item:cons:b} On définit ensuite nos trois points~$u',v',w'$ sur le cylindre~$\partial\BRz$ de la façon suivante : dès lors que~$\Cmin$ traverse le cylindre~$\BRz$ on peut prendre~$u'$ \marginnote{$u'$} le sommet par lequel il rentre dans~$\BRz$ et~$v'$ \marginnote{$v'$} le sommet par lequel il en ressort. Ces sommets sont distincts puisque~$\Cmin$ est auto-évitant. 
					
					Puisque~$z$ est dans~$U(\omega)$, par~\ref{item:P1}, il existe~$\pi'$ un chemin auto-évitant qui relie~$\overline{z+B_1}$ à~$S_n'$ ; on pose alors~$w'$ \marginnote{$w'$} le sommet de~$\partial\BRz$ par lequel~$\pi'$ sort pour la dernière fois de~$\BRz$ et~$\pi$\marginnote{$\pi$} le chemin qui relie alors~$w'$ à~$S_n'$ et qui ne rencontre~$\BRz$ qu'en~$w'$. On a alors~$w'$ distinct de~$u'$ et~$v'$ car sinon on aurait~$S_n'$ qui serait relié à~$S_{3n}$, en passant par~$\Cmin$ jusqu'à~$w'$ puis ensuite par~$\pi$.
					\item\label{item:cons:c} On retire toutes les arêtes de~$\omega$ qui sont dans~$\overline{B_3}(z)$ à l'exception de celles sur la frontière~$\partial\overline{B_3}(z)$ et qui sont dans les chemins~$\Cmin$ ou~$\pi$. ; de telle sorte que la seule façon d'entrer dans~$\BRz$ soit de passer par~$u', v'$ ou~$w'$
					\item\label{item:cons:d} Enfin, puisque~$u',v',w'$ sont trois sommets distincts de~$\partial\BRz$, on peut appliquer la remarque précédente qui nous assure l'existence de trois chemins auto-évitants disjoints~$\gamma_u,\gamma_v$ et~$\gamma_w$ \marginnote{$\gamma_u, \gamma_v, \gamma_w$} qui relient respectivement~$u,v,w$ à~$u'v',w'$ dans~$\BRz\priv\{z\}$.  On ouvre alors les arêtes~$(z,u), (z,v)$ et~$(z,w)$ ainsi que les chemins~$\gamma_u,\gamma_v$ et~$\gamma_w$. 
				\end{enumerate}
				On a construit~$\omega^{(z)}$ de telle sorte que~$S_{3n}$ et~$S_n'$ y soient reliés dans~$B_{3n}\cup B_n'$, ce qui nous permet de définir l'application suivante :
				\begin{align*}
					\Psi : \quad \mathcal{X} \cup \{\# U \geq t\} 	& \longrightarrow 
							\mathcal{P}\left(
								\relie[B_{3n}\cup B_n']{S_{3n}}{S_n'}		
							\right)\\
								\omega	& \longmapsto  \left\{\omega^{(z)}, z\in U(\omega)\right\}.
				\end{align*}
				On veut à présent utiliser le lemme~\ref{lem:interGluant}. Soit donc~$\omega^{(z)}$ une configuration à l'arrivée, on se rend compte dans un premier temps, par minimalité, que~$\Cmin[\omega^{(z)}]$ et~$\Cmin$ coïncident au moins jusqu'à~$u'$. Ensuite, et grâce au point~\ref{item:cons:c} de la construction précédente, on a~$u'$ de degré exactement~$2$ dans~$\omega$, et puisque l'on a vidé~$\BRz$, cela force~$\Cmin[\omega^{(z)}]$ à passer ensuite par~$\gamma_u$. Une fois arrivé en~$z$, on a deux possibilités : passer par~$v$ ou~$w$ ; mais puisque l'on a pris les sommets de telle sorte que~$(z,v)\prec (z,w)$ alors~$\Cmin[\omega^{(z)}]$ est forcé de passer par~$v$ et ensuite par~$\gamma_v$. Dès lors, la minimalité de~$\Cmin$ oblige~$\Cmin[\omega^{(z)}]$ à prendre le même chemin à la sortie de~$\BRz$.
				
				Pour résumer on a~$\Cmin[\omega^{(z)}]$ qui empreinte~$\Cmin$ à l'extérieur de~$\BRz$ ; alors que dans~$\BRz$ il empreinte~$\gamma_u$ de~$u'$ jusqu'à~$z$, puis~$\gamma_v$ de~$z$ jusqu'à~$v'$, où il rejoint~$\Cmin$. Aussi~$z$ est le seul sommet de~$\Cmin[\omega^{(z)}]$ qui soit relié à~$\overline{S_n'}$ sans utiliser aucune arête du chemin minimal~$\Cmin[\omega^{(z)}]$. Cela nous donne alors que les~$\omega^{(z)}$ sont tous distincts à~$\omega$ fixé. Dès lors, toutes les images~$\Psi(\omega)$ sont de cardinal au moins~$t$.
				
				Il nous reste maintenant à déterminer~$s$ tel que quelque soit l'image~$\omega^{(z)}$ que l'on prenne alors il existe un sous ensemble fini~$S$ de~$\Sk$ qui soit de cardinal au plus~$s$ et tel que l'on ait l'inclusion~$\{\omega\in\mathcal{X} \cup \{\# U \geq t\} ,\omega'\in\Psi(\omega)\} \subset \{\omega\in\mathcal{X} \cup \{\# U \geq t\} ,\omega_{|S^c}=\omega_{|S^c}'\}$. On prend~$s=84(k+1)+49 k$, le nombre d'arêtes dans~$\overline{B_3}(z)$, seul endroit de~$\omega^{(z)}$ qui diffère de~$\omega$. On peut enfin appliquer le lemme~\ref{lem:interGluant} qui nous donne la majoration :
				\[
					\Pp{\mathcal{X}\cap\{\# U \geq t\}} 
							\leq \frac{1}{t} \left(\frac{2}{\min(p,1-p)}\right)^s
									\Pp{\relie[B_{3n}\cup B_n']{S_{3n}}{S_n'}}.
				\]
				Or~$s$ est indépendant de~$t$, il suffit alors de prendre~$t$ assez grand pour conclure la démonstration de la proposition.
			\end{dem}
			On peut maintenant achever la démonstration du lemme gluant~(lemme~\ref{lem:gluant}). Fixons~$\epsilon$ un réel strictement positif, alors d'après la proposition~\ref{prop:Ugrand}, il existe~$t$ un entier naturel non nul tel que :
			\begin{equation}\label{eq:finalGluant:1}
					\Pp{\mathcal{X}\cap\{\#U \geq t\}} 
				\leq 
					\frac{\epsilon}{2}
					\cdot
					\Pp{\relie[B_{3n}\cup B_n']{S_{3n}}{S_n'}}.			
			\end{equation}
			On peut ensuite appliquer la proposition~\ref{prop:Upetit} à~$\epsilon/2$ et~$t$ : il existe un réel~$\delta$ strictement positif tel que lorsque la probabilité de l'événement~$\{\relie[B_n']{S_n'}{Y_n^-}\}\cap\{\relie[B_n']{S_n'}{Y_n^+}\}$ est plus grande que~$1-\delta$ on ait :
			\begin{equation}\label{eq:finalGluant:2}
				\Pp{\mathcal{X}\cap\{\#U < t\}}
				\leq
				\frac{\epsilon}{2}
				\cdot %ponctuation
			\end{equation}
			Dès lors, lorsque la probabilité de~$\{\relie[B_{3n}\cup B_n']{S_{3n}}{S_n'}\} \cap \{\relie[B_n']{S_n'}{Y_n^-}\} \cap \{\relie[B_n']{S_n'}{Y_n^+}\}$ est plus grande que~$1-\delta$, on peut utiliser les équations~\eqref{eq:finalGluant:1} et~\eqref{eq:finalGluant:2} et la décomposition de~$\mathcal{X}$ en~$(\mathcal{X}\cap\{\# U \geq t\}) \cup (\mathcal{X}\cap \{\# U < t\})$  pour obtenir~$\PpSP(\mathcal{X}) \leq \epsilon$ ; ce qui achève la démonstration du lemme gluant.

