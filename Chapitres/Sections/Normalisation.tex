\section{Renormalisation}
	\subsection{Un dernier calcul}
		On passe maintenant à la première étape de la renormalisation. Il s'agit de relier des ensembles du type~$S_{3n}$ éloignés de~$4n$. En appliquant l'inégalité de Harris-FKG dans la probabilité suivante, on arrive à une minoration exploitable à l'aide du lemme~\ref{lem:gluant}. On a en effet 
		\begin{align*}
			& \Pp{\relie[(2n, 0) + B_{6n}]{S_{3n}}{(4n, 0) + S_{3n}}}  \\
			& \geq \Pp{ \relie[B_{4n}]{S_{3n}}{S_n'} \cap \relie[(4n, 0) + B_{4n}]{S_n'}{(4n, 0) + S_{3n}} \cap \relie*[B_n']{S_n'}{\partial B_n'} } \\
			& \geq \Pp{ \relie[B_{4n}]{S_{3n}}{S_n'} \cap \relie[(4n, 0) + B_{4n}]{S_n'}{(4n, 0) + S_{3n}}} + \Pp{\relie*[B_n']{S_n'}{\partial B_n'}} \\ 
			& \geq \Pp{ \relie[B_{4n}]{S_{3n}}{S_n'} }^2 + \Pp{\relie*[B_n']{S_n'}{\partial B_n'}}  - 1
		\end{align*}
		Il est à noter que c'est l'unicité du chemin entre~$S_n'$ et~$\partial B_n'$ qui permet d'écrire la première inégalité. \\
		On conclût enfin à l'aide du lemme~\ref{lem:gluant} par le fait que la probabilité de~$\relie[B_{4n}]{S_{3n}}{S_n'}$ prend des valeurs arbitrairement proches de~$1$ et que la probabilité de~$\relie*[B_n']{S_n'}{\partial B_n'}$ tend vers~$1$. Ainsi
			\[
				\limsup_{n \to \infty} \Pp{\relie[(2n, 0) + B_{6n}]{S_{3n}}{(4n, 0) + S_{3n}}} = 1
			\]
	
	\subsection{Renormalisation}
	Ayant démontré le résutat précédent, il est naturel d'introduire le modèle suivant : on dit qu'une arrête~$(z_1, z_2)$ de~$4n\mathbb{Z}$ est ouverte si on a
		\begin{itemize}
			\item $\relie[R_n]{z_1 + S_{3n}}{z_2 + S_{3n}}$
			\item $\relie*[z_1 + B_{3n}]{z_1 + S_{3n}}{z_1 + \partial B_{3n}}$ et~$\relie*[z_2 + B_{3n}]{z_2 + S_{3n}}{z_2 + \partial B_{3n}}$ 
		\end{itemize}
	Il est facile de vérifier que ce modèle de percolation est~$4$-dépendant. On en déduit que si la probabilité qu'une arrête soit ouverte est assez proche de~$1$ alors la percolation a lieu. Or d'après le résultat précdent, et TRUCÀREMPLACERPARNUMEROD'EQUATION on peut rendre cette probabilité arbitrairement proche de~$1$. Fixons un~$n$ permettant la percolation, la probabilité de percolation ne dépendant que d'un nombre fini d'arrêtes, on a encore la percolation pour~$q$ suffisamment proche de~$p$.

	Ainsi,~$p$ est différente de~$p_c$ et il n'y a pas percolation au point critique, que qui prouve le théorème 1.
