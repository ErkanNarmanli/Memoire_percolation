\vspace*{0.3cm}
	Le modèle de percolation a été introduit par le mathématicien anglais John Hammerersley en 1957 dans son article \emph{Percolation Processes: Lower Bounds for the Critical Probability}~\cite{hammer}. En~1960, T.E. Harris démontre dans~\cite{harris} qu'il n'y a pas percolation dans le réseau~$\mathbb{Z}^2$ avec le paramètre~$1/2$. Plus tard en~1980, Harry Kesten démontre dans~\cite{kesten} que le le point critique de ce même réseau est plus petit que~$1/2$ ce qui démontre alors~$p_c=1/2$ et l'absence de percolation au point critique.
	
	En 1994 les mathématiciens Takashi Hara et Gordon Slade démontrent dans~\cite{mean} qu'il n'y a pas percolation au point critique dans~$\mathbb{Z}^n$ pour~$n\geq 19$. Bien l'on suspecte que le résultat soit vrai pour tout~$n$, le problème s'avère difficile, et montrer l'absence de percolation au point critique dans le réseau cubique~$\mathbb{Z}^3$ reste un domaine majeur de la recherche actuelle.
	
	Dans cet exposé, on s'intéresse aux \emph{tranches} du réseau cubique, à savoir les réseaux de la forme~$\mathbb{Z}^2\times\{0,...,k\}$. Par la suite, un tel réseau sera noté~$\Sk$. \marginnote{$\Sk$}
	Dans la suite, si~$p$ est un réel compris entre~$0$ et~$1$ on notera~$\PpSP$ \marginnote{$\PpSP$} la probabilité associée au modèle de percolation sur~$\Sk$ obtenu à partir de la Bernoulli de paramètre~$p$ sur les arrêtes. Si~$B$ est une parie de~$\mathbb{Z}^2$, on notera~$\overline{B}$ \marginnote{$\overline{B}$} le cylindre de base~$B$ donné par~$\overline{B}=B\times\{0,...,k\}$. 	
	Si~$B, X$ et~$Y$ sont trois sous-ensembles de~$\mathbb{Z}^2$ tels que~$X,Y\subset B$, alors on définit les deux événements suivants :
	\begin{itemize}
		\item[] $\relie[B]{X}{Y}$ \marginnote{$\relie[B]{X}{Y}$} : \og{} il existe un cluster ouvert infini dans $\overline{B}$ reliant $\overline{X}$ et $\overline{Y}$  \fg{}
		\item[] $\relie*[B]{X}{Y}$ \marginnote{$\relie*[B]{X}{Y}$} :  \og{} il existe un \emph{unique} cluster ouvert infini dans $\overline{B}$ reliant $\overline{X}$ et $\overline{Y}$  \fg{}
	\end{itemize}
	